\documentclass[11pt]{paper}
\usepackage{geometry}
\usepackage{hyperref}
\usepackage{xcolor}
\usepackage{amsmath}
\usepackage{natbib}
\geometry{
  top = 1in
  , bottom = 1in
  , left = 1in
  , right = 1in
  }
\hypersetup{
	colorlinks=true,
	linkcolor=blue,
	filecolor=magenta,
	urlcolor=cyan,
}

\begin{document}
\title{QMB 3311: Getting Started}
\author{Joshua L. Eubanks -- University of Central Florida -- Department of Economics}

\maketitle
\hrulefill

\tableofcontents

\section{Accounts}

There will be two places you will need an account; Webcourses and GitHub.

\subsubsection*{Webcourses}

Webcourses will be where you will submit links to all your assignments and exams. You should already know your credentials. If you do not know your credentials, here is how to find/reset them:

\begin{enumerate}
\item Navigate to \href{webcourses.ucf.edu}{webcourses.ucf.edu}
\item On the right hand side, there are grey boxes. You can look up your UCF NID or reset your password.
\end{enumerate}

\subsubsection*{GitHub}

GitHub is an excellent version control software used by many large companies. Think of it as a ``OneDrive/Google Drive" for programmers. It provides a history of all the saved changes you have made and allows for easy collaboration. 

To create an account:
\begin{enumerate}
\item Go to \href{https://github.com/signup?source=login}{https://github.com/signup?source=login}
\item Follow the prompts
\end{enumerate} 

\section{GitHub Desktop}

There are many other features of GitHub that allow you to make changes from the command line, however, GitHub Desktop avoids the command line. To install:

\begin{enumerate}
\item Download the file at: \href{https://desktop.github.com/}{https://desktop.github.com/}
\item Follow the steps, default settings for everything
\item Log in with your GitHub Account
\end{enumerate}

\section{Anaconda}

Anaconda is the most popular python distribution. It also installs many other packages by default, avoiding the requirement to install many of the packages manually. Additionally, it contains Spyder, a nice graphical user interface (GUI) that helps you navigate python easier. To install:

\begin{enumerate}
\item Go to: \href{https://www.anaconda.com/products/distribution}{https://www.anaconda.com/products/distribution}
\item Default settings for everything
\end{enumerate}

\end{document}